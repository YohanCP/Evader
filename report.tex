\documentclass[11pt,a4paper]{article}
%%%%%%%%%%%%%%%%%%%%%%%%% Credit %%%%%%%%%%%%%%%%%%%%%%%%

% template ini dibuat oleh martin.manullang@if.itera.ac.id untuk dipergunakan oleh seluruh sivitas akademik itera.

%%%%%%%%%%%%%%%%%%%%%%%%% PACKAGE starts HERE %%%%%%%%%%%%%%%%%%%%%%%%
\usepackage{graphicx}
\usepackage{caption}
\usepackage{microtype}
\captionsetup[table]{name=Tabel}
\captionsetup[figure]{name=Gambar}
\usepackage{tabulary}
\usepackage{minted}
% \usepackage{amsmath}
\usepackage{fancyhdr}
% \usepackage{amssymb}
% \usepackage{amsthm}
\usepackage{placeins}
% \usepackage{amsfonts}
\usepackage{graphicx}
\usepackage[all]{xy}
\usepackage{tikz}
\usepackage{verbatim}
\usepackage[left=2cm,right=2cm,top=3cm,bottom=2.5cm]{geometry}
\usepackage{hyperref}
\hypersetup{
    colorlinks,
    linkcolor={red!50!black},
    citecolor={blue!50!black},
    urlcolor={blue!80!black}
}
\usepackage{caption}
\usepackage{subcaption}
\usepackage{multirow}
\usepackage{psfrag}
\usepackage[T1]{fontenc}
\usepackage[scaled]{beramono}
% Enable inserting code into the document
\usepackage{listings}
\usepackage{color}

\definecolor{dkgreen}{rgb}{0,0.6,0}
\definecolor{gray}{rgb}{0.5,0.5,0.5}
\definecolor{mauve}{rgb}{0.58,0,0.82}

\lstset{frame=tb,
  language=Java,
  aboveskip=3mm,
  belowskip=3mm,
  showstringspaces=false,
  columns=flexible,
  basicstyle={\small\ttfamily},
  numbers=none,
  numberstyle=\tiny\color{gray},
  keywordstyle=\color{blue},
  commentstyle=\color{dkgreen},
  stringstyle=\color{mauve},    
  breaklines=true,
  breakatwhitespace=true,
  tabsize=3
}
\renewcommand{\lstlistingname}{Kode}
%%%%%%%%%%%%%%%%%%%%%%%%% PACKAGE ends HERE %%%%%%%%%%%%%%%%%%%%%%%%


%%%%%%%%%%%%%%%%%%%%%%%%% Data Diri %%%%%%%%%%%%%%%%%%%%%%%%
\newcommand{\student}{\textbf{Isi Nama Di Sini (Dan Nim Di Sini)}}
\newcommand{\course}{\textbf{Nama Mata Kuliah (Kode Mata Kuliah)}}
\newcommand{\assignment}{\textbf{xxx}}

%%%%%%%%%%%%%%%%%%% using theorem style %%%%%%%%%%%%%%%%%%%%
\newtheorem{thm}{Theorem}
\newtheorem{lem}[thm]{Lemma}
\newtheorem{defn}[thm]{Definition}
\newtheorem{exa}[thm]{Example}
\newtheorem{rem}[thm]{Remark}
\newtheorem{coro}[thm]{Corollary}
\newtheorem{quest}{Question}[section]
%%%%%%%%%%%%%%%%%%%%%%%%%%%%%%%%%%%%%%%%
\usepackage{lipsum}%% a garbage package you don't need except to create examples.
\usepackage{fancyhdr}
\pagestyle{fancy}
\lhead{Nama Mahasiswa di Header (Nim Mahasiswa di Header)}
\rhead{ \thepage}
\cfoot{\textbf{Judul Tugas diketik di sini}}
\renewcommand{\headrulewidth}{0.4pt}
\renewcommand{\footrulewidth}{0.4pt}

%%%%%%%%%%%%%%  Shortcut for usual set of numbers  %%%%%%%%%%%

\newcommand{\N}{\mathbb{N}}
\newcommand{\Z}{\mathbb{Z}}
\newcommand{\Q}{\mathbb{Q}}
\newcommand{\R}{\mathbb{R}}
\newcommand{\C}{\mathbb{C}}
\setlength\headheight{14pt}

%%%%%%%%%%%%%%%%%%%%%%%%%%%%%%%%%%%%%%%%%%%%%%%%%%%%%%%555
\begin{document}
\thispagestyle{empty}
\begin{center}
	\includegraphics[scale = 0.15]{Figure/ifitera-header.png}
	\vspace{0.1cm}
\end{center}
\noindent
\rule{17cm}{0.2cm}\\[0.3cm]
Nama: \student \hfill Tugas Ke: \assignment\\[0.1cm]
Mata Kuliah: \course \hfill Tanggal: Tanggalnya\\
\rule{17cm}{0.05cm}
\vspace{0.1cm}



%%%%%%%%%%%%%%%%%%%%%%%%%%%%%%%%%%%%%%%%%%%%% BODY DOCUMENT %%%%%%%%%%%%%%%%%%%%%%%%%%%%%%%%%%%%%%%%%%%%%
\section{Deskripsi Projek}
Proyek ini adalah sebuah permainan tembak-menembak interaktif untuk dua pemain yang dikendalikan dengan kedipan mata dan gestur tangan menggunakan webcam. Pemain dapat menembakkan "peluru" dengan berkedip dan mengaktifkan "perisai" dengan gestur jempol.

\section{Teknologi}
Dalam projek ini digunakan beberapa library dari python yang esensial dalam menjalankan program. Berikut merupakan library yang digunakan :
\begin{itemize}
    \item cv2 (OpenCV): Digunakan untuk mengambil video dari webcam, memproses frame video (misalnya membalik atau mengubah ukuran), dan menampilkan antarmuka permainan.
    \item mediapipe: Kerangka kerja lintas platform untuk membangun alur kerja pembelajaran mesin, digunakan di sini untuk deteksi dan pelacakan landmark wajah (untuk mata dan hidung) dan tangan.
    \item numpy: Library untuk komputasi numerik, khususnya untuk operasi array, digunakan dalam perhitungan rasio aspek mata dan penanganan mask gambar.
    \item threading: Library untuk menjalankan operasi secara bersamaan dalam program (misalnya, menonaktifkan perisai setelah beberapa detik).
    \item time: Library untuk fungsi terkait waktu, seperti mengukur cooldown kedipan dan durasi perisai.
    \item overlay (modul kustom): Modul ini berisi fungsi untuk menempatkan gambar (seperti sprite pemain, peluru, dan perisai) di atas frame video secara transparan.
    \item utils (modul kustom): Modul ini berisi fungsi-fungsi pembantu seperti perhitungan rasio aspek mata, deteksi gestur tangan, dan penggambaran healthbar.
\end{itemize}

\section{Cara Kerja}
Cara kerja pada projek Final Project ini adalah sebagai berikut:
\begin{itemize}
    \item Sistem menggunakan kamera webcam untuk menangkap video secara langsung.
    \item Sistem menggunakan library MediaPipe Face Mesh untuk mendeteksi landmark wajah, khususnya di sekitar mata, untuk menghitung Eye Aspect Ratio (EAR).
    \item Berdasarkan EAR, sistem mendeteksi kedipan mata pengguna. Jika mata terdeteksi berkedip dan memenuhi kondisi cooldown, proyektil ditembakkan dari posisi wajah pemain.
    \item Sistem juga menggunakan library MediaPipe Hands untuk mendeteksi tangan dan melacak posisi landmark jari.
    \item Berdasarkan posisi landmark tangan, sistem mendeteksi gestur tangan tertentu (misalnya, thumbs up).
    \item Jika gestur thumbs up terdeteksi, shield untuk pemain yang bersangkutan akan diaktifkan untuk durasi tertentu, memberikan perlindungan dari proyektil lawan.
    \item Posisi hidung dari deteksi face mesh digunakan untuk menentukan posisi horizontal dan vertikal karakter pemain di layar.
    \item Proyektil bergerak melintasi layar, dan sistem akan mendeteksi tabrakan antara proyektil dan pemain lawan. Jika terjadi tabrakan dan shield tidak aktif, health pemain akan berkurang.
    \item Healthbar pemain ditampilkan di layar untuk menunjukkan kondisi health mereka secara real-time.
\end{itemize}

\section{Penjelasan Kode Program}
Kode program terbagi menjadi tiga file: \textbf{main.py}, \textbf{overlay.py}, dan \textbf{utils.py}. Berikut adalah penjelasan dari masing-masing bagian kode:
\subsection{main.py}
\textbf{main.py} berisi kode yang menjadi fondasi utama program, pada kode ini program dijalankan kemudian akan memanggil fungsi yang dibutuhkan dari file pythona lainnya

\begin{lstlisting}[language=Python, caption=Main.py]
import cv2
import mediapipe as mp
import numpy as np
import threading
import time
from overlay import overlay_transparent
from utils import eye_aspect_ratio, detect_hand_gesture, draw_healthbar

# Inisialisasi Face Mesh dan Hand Detection
mp_face_mesh = mp.solutions.face_mesh
face_mesh = mp_face_mesh.FaceMesh(max_num_faces=2, refine_landmarks=True)
mp_hands = mp.solutions.hands
hands = mp_hands.Hands(max_num_hands=2)

# Load sprite player, peluru, dan shield
player_img = cv2.imread('assets/player_dummy.png', cv2.IMREAD_UNCHANGED)
ammo_img = cv2.imread('assets/ammo_dummy.png', cv2.IMREAD_UNCHANGED)
shield_img = cv2.imread('assets/player_dummy.png', cv2.IMREAD_UNCHANGED)

if player_img is None or ammo_img is None or shield_img is None:
    raise FileNotFoundError("Gambar tidak ditemukan.")
if player_img.shape[2] < 4 or ammo_img.shape[2] < 4 or shield_img.shape[2] < 4:
    raise ValueError("Gambar tidak memiliki alpha channel.")

# Indeks landmark mata
LEFT_EYE_IDX = [362, 385, 387, 263, 373, 380]
RIGHT_EYE_IDX = [33, 160, 158, 133, 153, 144]

# Landmark hidung untuk posisi player
NOSE_IDX = 1

# Threshold dan cooldown
BLINK_THRESHOLD = 0.15
OPEN_THRESHOLD = 0.25
BLINK_COOLDOWN = 1.0

# Ukuran player
PLAYER_W = 100
PLAYER_H = int(PLAYER_W * player_img.shape[0] / player_img.shape[1])

# Healthbar awal
health_player1 = 100
health_player2 = 100

# Status shield
shield_active_player1 = False
shield_active_player2 = False

def activate_shield(player_id):
    #Activate shield for the specified player.

    #Args:
    #    player_id (str): "Player 1" or "Player 2".
    global shield_active_player1, shield_active_player2
    if player_id == "Player 1":
        shield_active_player1 = True
        print("Player 1 activated SHIELD!")
    elif player_id == "Player 2":
        shield_active_player2 = True
        print("Player 2 activated SHIELD!")

    # Matikan shield setelah beberapa detik
    def deactivate_shield():
        nonlocal player_id
        time.sleep(5)  # Shield aktif selama 5 detik
        if player_id == "Player 1":
            shield_active_player1 = False
        elif player_id == "Player 2":
            shield_active_player2 = False
        print(f"{player_id} shield deactivated.")

    threading.Thread(target=deactivate_shield).start()

def main():
    # Main function to run the interactive blink-based shooting game with two players.
    # Buka kamera
    cap = cv2.VideoCapture(0)
    if not cap.isOpened():
        raise IOError("Tidak dapat membuka kamera.")

    # Inisialisasi variabel
    last_blink_time = [0, 0]
    eye_ready_to_blink = [0, 0]
    projectiles = []
    health_player1 = 100
    health_player2 = 100

    while True:
        ret, frame = cap.read()
        if not ret:
            break

        frame = cv2.flip(frame, 1)
        ih, iw = frame.shape[:2]
        rgb_frame = cv2.cvtColor(frame, cv2.COLOR_BGR2RGB)
        results_face = face_mesh.process(rgb_frame)
        results_hands = hands.process(rgb_frame)
        current_time = time.time()

        # Reset proyektil yang keluar dari layar
        new_projectiles = []
        for proj in projectiles:
            proj['x'] += 10 if proj['player'] == "Player 1" else -10
            if 0 <= proj['x'] <= iw:
                new_projectiles.append(proj)
        projectiles = new_projectiles

        if results_face.multi_face_landmarks:
            for idx, face_landmarks in enumerate(results_face.multi_face_landmarks[:2]):
                # Deteksi kedipan mata
                left_ear = eye_aspect_ratio(face_landmarks.landmark, LEFT_EYE_IDX, iw, ih)
                right_ear = eye_aspect_ratio(face_landmarks.landmark, RIGHT_EYE_IDX, iw, ih)
                avg_ear = (left_ear + right_ear) / 2.0

                nose = face_landmarks.landmark[NOSE_IDX]
                player_x = int(nose.x * iw) - PLAYER_W // 2
                player_y = int(nose.y * ih) - PLAYER_H // 2

                blinking = avg_ear < BLINK_THRESHOLD
                eyes_open = avg_ear > OPEN_THRESHOLD

                if eyes_open:
                    eye_ready_to_blink[idx] = 1

                if blinking and eye_ready_to_blink[idx] == 1 and current_time - last_blink_time[idx] >= BLINK_COOLDOWN:
                    # Tembakkan peluru
                    ammo_w = 30
                    ammo_h = int(ammo_w * ammo_img.shape[0] / ammo_img.shape[1])
                    ammo_start_x = player_x + PLAYER_W if idx == 0 else player_x - ammo_w
                    ammo_start_y = player_y + PLAYER_H // 2 - ammo_h // 2
                    projectiles.append({
                        'x': ammo_start_x,
                        'y': ammo_start_y,
                        'w': ammo_w,
                        'h': ammo_h,
                        'player': f"Player {idx + 1}"
                    })
                    last_blink_time[idx] = current_time
                    eye_ready_to_blink[idx] = 0

                # Gambar player
                frame = overlay_transparent(frame, player_img, player_x, player_y, (PLAYER_W, PLAYER_H))

                # Gambar healthbar
                health = health_player1 if idx == 0 else health_player2
                frame = draw_healthbar(frame, f"Player {idx + 1}", health, player_x, player_y - 20)

        # Deteksi gestur tangan
        if results_hands.multi_hand_landmarks:
            for hand_landmarks in results_hands.multi_hand_landmarks:
                gesture = detect_hand_gesture(hand_landmarks)
                if gesture == "thumbs_up":
                    # Aktifkan shield untuk pemain yang melakukan gestur
                    if hand_landmarks.landmark[0].x < 0.5:  # Pemain di kiri frame
                        activate_shield("Player 1")
                    else:  # Pemain di kanan frame
                        activate_shield("Player 2")

        # Gambar proyektil dan deteksi tabrakan
        for proj in projectiles:
            frame = overlay_transparent(frame, ammo_img, proj['x'], proj['y'], (proj['w'], proj['h']))

            # Deteksi tabrakan dengan player
            for idx, face_landmarks in enumerate(results_face.multi_face_landmarks[:2]):
                nose = face_landmarks.landmark[NOSE_IDX]
                player_x = int(nose.x * iw) - PLAYER_W // 2
                player_y = int(nose.y * ih) - PLAYER_H // 2

                if (player_x < proj['x'] < player_x + PLAYER_W and
                    player_y < proj['y'] < player_y + PLAYER_H and
                    proj['player'] != f"Player {idx + 1}" and
                    not (shield_active_player1 if idx == 0 else shield_active_player2)):
                    if idx == 0:
                        health_player1 -= 10
                    else:
                        health_player2 -= 10
                    projectiles.remove(proj)

        # Gambar shield jika aktif
        for idx, face_landmarks in enumerate(results_face.multi_face_landmarks[:2]):
            nose = face_landmarks.landmark[NOSE_IDX]
            player_x = int(nose.x * iw) - PLAYER_W // 2
            player_y = int(nose.y * ih) - PLAYER_H // 2

            if idx == 0 and shield_active_player1:
                frame = overlay_transparent(frame, shield_img, player_x, player_y, (PLAYER_W, PLAYER_H))
            elif idx == 1 and shield_active_player2:
                frame = overlay_transparent(frame, shield_img, player_x, player_y, (PLAYER_W, PLAYER_H))

        # Tampilkan frame
        cv2.imshow("Two Player Blink Shot", frame)
        if cv2.waitKey(5) & 0xFF == 27:  # Tekan ESC untuk keluar
            break

    cap.release()
    cv2.destroyAllWindows()

if __name__ == "__main__":
    main()
\end{lstlisting}

\subsection{overlay.py}
\textbf{overlay.py} berisi kode untuk ...
\begin{lstlisting}[language=Python, caption=Main.py]
import cv2
import numpy as np

def overlay_transparent(bg, overlay, x, y, overlay_size=None):
    bg = bg.copy()
    if overlay_size:
        overlay = cv2.resize(overlay, overlay_size, interpolation=cv2.INTER_AREA)
    h, w = overlay.shape[:2]
    if x + w > bg.shape[1] or y + h > bg.shape[0] or x < 0 or y < 0:
        return bg
    overlay_img = overlay[:, :, :3]
    mask = overlay[:, :, 3] / 255.0
    mask = np.stack([mask] * 3, axis=-1)
    roi = bg[y:y+h, x:x+w]
    blended = (1.0 - mask) * roi + mask * overlay_img
    bg[y:y+h, x:x+w] = blended.astype(np.uint8)
    return bg
\end{lstlisting}

\subsection{utils.py}
\textbf{utils.py}, seperti namanya berisi utilitas yang dibutuhkan program seperti aspek rasio mata, mendeteksi gesture tangan dan menggambar healthbar untuk player
\begin{lstlisting}
[language=Python, caption=Main.py]
import numpy as np
import cv2

def eye_aspect_ratio(landmarks, eye_indices, img_w, img_h):
    p = [landmarks[i] for i in eye_indices]
    p = [(int(pt.x * img_w), int(pt.y * img_h)) for pt in p]
    A = np.linalg.norm(np.array(p[1]) - np.array(p[5]))
    B = np.linalg.norm(np.array(p[2]) - np.array(p[4]))
    C = np.linalg.norm(np.array(p[0]) - np.array(p[3]))
    ear = (A + B) / (2.0 * C)
    return ear

def detect_hand_gesture(hand_landmarks):
    thumb_tip = hand_landmarks.landmark[4]
    index_tip = hand_landmarks.landmark[8]
    middle_tip = hand_landmarks.landmark[12]

    # Contoh: Deteksi thumbs up
    if thumb_tip.y < index_tip.y and thumb_tip.y < middle_tip.y:
        return "thumbs_up"

    return None

def draw_healthbar(frame, player_id, health, x, y, w=100, h=10):
    # Draw background of the healthbar
    cv2.rectangle(frame, (x, y), (x + w, y + h), (0, 0, 0), 2)
    # Draw filled part of the healthbar
    fill_width = int((health / 100) * w)
    cv2.rectangle(frame, (x, y), (x + fill_width, y + h), (0, 255, 0), -1)
    # Add player label
    cv2.putText(frame, player_id, (x, y - 5), cv2.FONT_HERSHEY_SIMPLEX, 0.5, (255, 255, 255), 1)
    return frame
\end{lstlisting}





\section{Memuat Tabel}
Bagian ini adalah bagian yang menurut saya cukup sulit. Namun anda dapat menggunakan bantuan dari table designer yang ada di internet, misalnya \href{https://www.tablesgenerator.com}{TablesGenerator} atau \href{https://www.latex-tables.com}{Latex-Tables}. Ada juga plugin untuk microsoft excel bernama \href{https://ctan.org/tex-archive/support/excel2latex?lang=en}{CTAN}, namun saya jarang mempergunakannya. Contoh tabel dapat dilihat pada Tabel \ref{tab-contoh} berikut.

\begin{table}[h]
\caption{Contoh Tabel}
\label{tab-contoh}
\centering
\resizebox{12cm}{!}{%
\begin{tabular}{ccccccccc}
\hline
\multirow{2}{*}{\textbf{Exp}} & \multirow{2}{*}{\textbf{Mask}} & \multicolumn{3}{c}{\textbf{GT}}            & \multicolumn{3}{c}{\textbf{Proposed}}      & \multirow{2}{*}{\textbf{RMSE}} \\ \cline{3-8}
                              &                                & \textbf{Avg} & \textbf{Max} & \textbf{Min} & \textbf{Avg} & \textbf{Max} & \textbf{Min} &                                \\ \hline
\multirow{2}{*}{1}            & Yes                            & 89.60        & 114.84        & 70.31        & 89.14        & 118.45        & 68.24        & 3.66                           \\
                              & No                             & 90.55        & 112.50        & 75.00        & 89.03        & 109.75        & 71.35        & 3.60                            \\ \hline
\multirow{2}{*}{2}            & Yes                            & 109.84        & 125.98       & 98.44       & 108.62       & 121.30       & 98.26      & 4.04                           \\ 
                              & No                             & 106.62       & 123.44        & 96.09       & 106.48       & 122.19       & 93.37        & 3.95                           \\ \hline
3                             & Yes                            & 74.42        & 94.92       & 62.99        & 73.49       & 102.32        & 60.43       & 3.27                         \\ \hline
Mean                          &                                & 90.61         & 114.34        & 80.57        & 89.70       & 114.80       & 78.33        & 3.63                           \\ \hline
\end{tabular}
}
\end{table}

\section{Referensi dan Daftar Pustaka}
Ini bagian yang sedikit \textit{tricky}. Anda harus memasukkan daftar pustaka anda ke sebuah file berekstensi .bib di bagian kiri dari overleaf ini. Konten dot bib ini dapat anda export dengan mudah, entah itu dari google scholar, mendeley, ataupun manajemen sitasi lainnya.\\\\
Cara penggunaanya pun cukup mudah. Misalnya saat ini saya ingin mensitasi salah satu dokumen yang ada, misalnya wikipedia, saya cukup menuliskan $\backslash${\tt{cite}} yang berisikan cite-key dari entri yang ada di file.bib \cite{Wikipedia_contributors2021-bb}. Contoh lain menuliskan sitasi adalah sebagai berikut \cite{Name2018-hd}.

\newpage
\bibliographystyle{IEEEtran}
\bibliography{Referensi}
\end{document}